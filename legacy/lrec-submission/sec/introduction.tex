\section{Introduction}
Automatic language identification is a challenging problem and especially discriminating between closely related languages is one of the main bottlenecks of state-of-the-art language identification systems.\cite{DSL2014}\\

This paper presents a machine learning approach for automatic language identification for the Nordic languages. Concretely we will focus on discrimination between the six Nordic languages: Danish, Swedish, Norwegian (Nynorsk), Norwegian (Bokmål), Faroese and Icelandic.\\

This papers explore different ways of extracting features from a corpus of raw text data consisting of Wikipedia summaries in respective languages and evaluates the performance of a selection of machine learning models.\\

Concretely we will compare the performance of classic machine learning models such as Logistic Regression, Naive Bayes, Support vector machine, and K nearest Neighbors with more contemporary neural network approaches such as Multilayer Perceptrons (MLP) and Convolutional Neural Networks (CNNs).\\

After evaluating these models on the Wikipedia data set we will continue to evaluate the best models on a data set from a different domain in order to investigate how well the models generalize when classifying sentences from a different domain.
